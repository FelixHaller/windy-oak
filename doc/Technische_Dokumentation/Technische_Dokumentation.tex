\documentclass[150]{HSMW-Thesis}

\Art{Beleg Verteilte Systeme}


\Vorname{Felix Haller und }
\Nachname{Konstantin Lorenz}


\Thema{}
\Unterthema{}

\Studiengang{Angewandte Informatik Profilrichtung IT-Sicherheit}
\Seminargruppe{IF13wI-B}
\Fakultaet{Angewandte Computer- und Biowissenschaften}

\Erstpruefer{}
\Zweitpruefer{}

\Datum{}

\Tag{}
\Monat{}
\Jahr{}

\Anlagen{}
\Copyright{}
\Textsatz{}
\Druck{}
\Verlag{}
\ISBN{}

\begin{document}


\begin{Vorwort}
% Vorwort
\end{Vorwort}

\Hauptteil
% Diese Anweisung nicht loeschen!

\chapter{Aufgabenstellung}
\chapter{Motivation}
\chapter{Installations- und Startanleitung}
\chapter{REST Ressourcen im Überblick}
Hier soll eine "Ubersichtsgrafik hin.
\chapter{REST-Schnittstellen}
 Schnittstellen im einzelnen dokumentieren:
 Welcher Dienst wird bereit gestellt? Was passiert?
 Welche Aufrufe sind dazu notwendig?
 Welche Parameter m"ussen "ubergeben werden? (obligatorisch, verpflichtend)
 Welche Status-Codes sind zu erwarten?
 Welche Ergebnisse werden zur"uckgegeben? 
\chapter{Datenmodell und Persistierung}
\chapter{Client}
Welche Klassen wurden zum Abkapseln des Webservice erstellt und welche Methoden
stellen diese Bereit? Kurz Methodenfunktionalität umreißen.
Beschreibung des Ablaufs des Testszenarios inkl. zu erwartender Zwischenergebnisse
\chapter{Zusammenfassung}
\chapter{Fazit}
\chapter{Arbeitsverteilung}
Wer hat was gemacht.
%\section{}

\Anhang

%\chapter{}

\begin{thebibliography}{99}
\bibitem{} 
\end{thebibliography}

\end{document}

%%% Local Variables:
%%% mode: latex
%%% TeX-master: t
%%% End:
