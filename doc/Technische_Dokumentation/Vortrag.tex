\documentclass{beamer}

\usepackage{tikz}
\usepackage[utf8]{inputenc}
\usepackage{amsmath}
\usepackage{commath}
\usepackage{amssymb}
\usepackage{amsfonts}
\usepackage{epstopdf}
\usepackage{graphicx}
\usepackage{graphics}
\usepackage{listings}

\usetheme[CB,contents,colthm,colmath]{Mittweida}

\author{Felix Haller, Konstantin Lorenz}
\title{REST Beleg}
\subtitle{Projektverwaltung für Studenten und Professoren}
\institute{IF13wI-B}
\date{}

\begin{document}

	\maketitle

	\nextframenocontents

	\section*{Inhalt}
	\begin{frame}{Inhalt}
		\tableofcontents
	\end{frame}
	\section{Aufgabenstellung}
	
	\section{Motivation}
		\begin{frame}{Motivation}
		\end{frame}
		
	\section{Bedienkonzept}
		\begin{frame}{Bedienkonzept}
			Moqups mit Erklärung.
			
		\end{frame}
	\section{Übersicht REST Ressourcen}
		\begin{frame}{Übersicht REST Ressourcen}
			
		\end{frame}
	\section{Testszenario}
		\begin{frame}{Testszenario}
			Hier wird der Client beschrieben.
		\end{frame}
		
	\section{Live Aufrufe von einigen REST Befehlen}
		\begin{frame}{Live REST Abfragen}
			
		\end{frame}
		
	\section{Fazit oder Zusammenfassung oder so}
		\begin{frame}{Fazit oder Zusammenfassung oder so}
			
		\end{frame}
	
	
\end{document}

