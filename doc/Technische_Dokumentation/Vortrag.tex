\documentclass{beamer}

\usepackage{tikz}
\usepackage[utf8]{inputenc}
\usepackage{amsmath}
\usepackage{commath}
\usepackage{amssymb}
\usepackage{amsfonts}
\usepackage{epstopdf}
\usepackage{graphicx}
\usepackage{graphics}
\usepackage{listings}

\usetheme[CB,contents,colthm,colmath]{Mittweida}

\author{Felix Haller}
\title{REST Beleg}
\subtitle{Projektverwaltung für Studenten und Professoren}
\institute{IF13wI-B}
\date{}

\begin{document}

	\maketitle

	\nextframenocontents

	\section*{Inhalt}
	\begin{frame}{Inhalt}
		\tableofcontents
	\end{frame}
	\section{Aufgabenstellung}
	
	\section{Motivation}
		\begin{frame}{Hashfunktionen}
		\end{frame}
		\begin{frame}[t]{Definition einer Hashfunktion}	
		\end{frame}
	\section{Übersicht REST Ressourcen}
	\section{Testszenario}
		\begin{frame}{Testszenario}
			Hier wird der Client beschrieben.
		\end{frame}
		
	\section{Live Aufrufe von einigen REST Befehlen}
	
	
\end{document}

